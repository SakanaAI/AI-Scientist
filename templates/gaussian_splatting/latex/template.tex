\documentclass{article} % For LaTeX2e
\usepackage{iclr2024_conference,times}

\usepackage[utf8]{inputenc} % allow utf-8 input
\usepackage[T1]{fontenc}    % use 8-bit T1 fonts
\usepackage{hyperref}       % hyperlinks
\usepackage{url}            % simple URL typesetting
\usepackage{booktabs}       % professional-quality tables
\usepackage{amsfonts}       % blackboard math symbols
\usepackage{nicefrac}       % compact symbols for 1/2, etc.
\usepackage{microtype}      % microtypography
\usepackage{titletoc}

\usepackage{subcaption}
\usepackage{graphicx}
\usepackage{amsmath}
\usepackage{multirow}
\usepackage{color}
\usepackage{colortbl}
\usepackage{cleveref}
\usepackage{algorithm}
\usepackage{algorithmicx}
\usepackage{algpseudocode}

\DeclareMathOperator*{\argmin}{arg\,min}
\DeclareMathOperator*{\argmax}{arg\,max}

\graphicspath{{../}} % To reference your generated figures, see below.
\begin{filecontents}{references.bib}
@article{lu2024aiscientist,
  title={The {AI} {S}cientist: Towards Fully Automated Open-Ended Scientific Discovery},
  author={Lu, Chris and Lu, Cong and Lange, Robert Tjarko and Foerster, Jakob and Clune, Jeff and Ha, David},
  journal={arXiv preprint arXiv:2408.06292},
  year={2024}
}

@article{kerbl3Dgaussians,
      author={Kerbl, Bernhard and Kopanas, Georgios and Leimk{\"u}hler, Thomas and Drettakis, George},
      title={3D Gaussian Splatting for Real-Time Radiance Field Rendering},
      journal={ACM Transactions on Graphics},
      number={4},
      volume={42},
      month={July},
      year={2023},
      url={https://repo-sam.inria.fr/fungraph/3d-gaussian-splatting/}
}

\end{filecontents}

\title{TITLE HERE}

\author{LLM\\
Department of Computer Science\\
University of LLMs\\
}

\newcommand{\fix}{\marginpar{FIX}}
\newcommand{\new}{\marginpar{NEW}}

\begin{document}

\maketitle

\begin{abstract}
ABSTRACT HERE
\end{abstract}

\section{Introduction}
\label{sec:intro}
INTRO HERE

\section{Related Work}
\label{sec:related}
RELATED WORK HERE

\section{Background}
\label{sec:background}
BACKGROUND HERE

\section{Method}
\label{sec:method}
METHOD HERE

\section{Experimental Setup}
\label{sec:experimental}
EXPERIMENTAL SETUP HERE

\section{Results}
\label{sec:results}
RESULTS HERE

% EXAMPLE FIGURE: REPLACE AND ADD YOUR OWN FIGURES / CAPTIONS
\begin{figure}[h]
  \centering
  \includegraphics[width=\textwidth]{south-building_l1.png}
  \caption{PLEASE FILL IN CAPTION HERE}
  \label{fig:seir_plot}
\end{figure}

\begin{figure}[h]
  \centering
  \includegraphics[width=\textwidth]{south-building_l1.png}
  \caption{PLEASE FILL IN CAPTION HERE}
  \label{fig:seir_plot}
\end{figure}

\begin{figure}[h]
  \centering
  \includegraphics[width=\textwidth]{south-building_images_comparison.png}
  \caption{PLEASE FILL IN CAPTION HERE}
  \label{fig:seir_plot}
\end{figure}

\section{Conclusions and Future Work}
\label{sec:conclusion}
CONCLUSIONS HERE

This work was generated by \textsc{The AI Scientist} \citep{lu2024aiscientist}.

\bibliographystyle{iclr2024_conference}
\bibliography{references}

\end{document}
